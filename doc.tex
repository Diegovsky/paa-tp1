\documentclass{article}

\title{Titulo}
\date{Maio de 2023}
\author{Diego Augusto}

% Syntax highlighting:
% \usepackage{minted}
% Imagens:
% \usepackage{graphicx}
% \graphicspath{ {./images/} }
\usepackage{indentfirst}
\setlength{\parindent}{1.5cm}

\begin{document}
    \pagenumbering{arabic}

    \maketitle
    \newpage
    \section{Introdução}
    O problema proposto para solucionar reside em um contexto de reinos
    encantados, onde deseja-se encontrar os $K$ caminhos mínimos saindo do reino Mysthollow até o reino de Luminae. Entre esses
    reinos existem $V$ cidades, e os caminhos entre as cidades contêm desafios. Este
    problema é bem conhecido na literatura e pode ser representado por um grafo.
    Portanto, aplicaremos uma técnica para encontrar o caminho mínimo (ou custo
    mínimo) em um grafo ponderado, onde os vértices são as cidades, tendo a
    cidade inicial como o reino de Mysthollow e a cidade final como o reino de
    Luminae. As arestas representam as ligações entre as cidades, e os desafios
    entre as cidades serão representados pelos pesos. Assim, resolver este
    problema significa determinar o caminho entre dois vértices com o custo
    mínimo, ou seja, com o menor tempo de viagem.

    Os dados para a construção do grafo são fornecidos pelo usuário através de um
    arquivo de entrada, e a solução do problema será apresentada em um arquivo de
    saída. O principal foco deste trabalho é o desenvolvimento e análise de
    algoritmos que solucionam o problema, levando em consideração a eficiência deles
    nos diversos contextos em que o problema pode estar inserido. Portanto, durante
    a execução do trabalho, foram analisados mais de um algoritmo para encontrar
    caminhos mínimos, pois a eficiência de alguns algoritmos é comprometida em
    certos contextos possíveis.

    Diante disso, é necessário desenvolver um programa que, além de encontrar a
    solução, preocupa-se com a otimização e facilite a manutenção por meio de boas
    práticas de programação e uma documentação adequada, como será demonstrado nas
    próximas seções.
    \section{Desenvolvimento do Projeto}
    Para representar o grafo, utilizamos a representação de lista de adjacência.
    Uma lista de adjacência consiste em uma lista de tamanho $|V|$, onde cada
    elemento guarda uma lista encadeada contendo as arestas do nó naquele
    índice. Ou seja: as arestas que saem do nó $v$ ficam guardadas no índice
    $v-1$ da lista de adjacência.

    Fizemos essa escolha pois é uma representação relativamente simples de se
    implementar, além de performar bem em grafos esparsos e razoavelmente em
    grafos densos se comparado à uma matriz de adjacência.

    \textbf{Imagem de uma lista de adjacencia}

    Para desenvolver a solução, optamos pelo algoritmo de Eppstein. Fizemos essa
    escolha pensando em várias características do problema, como a quantidade de
    nós, arestas e o valor de $K$. Isso se deve ao fato de que os valores de $K$
    não excedem 10, tornando o algoritmo bem rápido. O motivo será explicado nas
    próximas seções.

\end{document}
